% !TeX root = RJwrapper.tex
\title{A clustering algorithm to organize satellite hotspots data for the
purpose of tracking bushfires remotely}
\author{by Weihao Li, Emily Dodwell, Dianne Cook}

\maketitle

\abstract{%
An abstract of less than 150 words.
}

\hypertarget{introduction}{%
\subsection{Introduction}\label{introduction}}

\begin{itemize}
\tightlist
\item
  What is the data, generic structure
\item
  Lit review: Spatio-temporal clustering. Algorithms for tracking
  movement.
\item
  Bushfire literature review?
\end{itemize}

\hypertarget{algorithm}{%
\subsection{Algorithm}\label{algorithm}}

\hypertarget{data-pre-processing}{%
\subsubsection{Data pre-processing}\label{data-pre-processing}}

To track bushfires in Australia remotely, we used hotspots data taken
from the Himawari-8 satellite. The hotspots data is available on the
JAXA FTP site in CSV file format, and only the data during October 2019
to March 2020 was downloaded. It contains records of 1989572 hotspots
for 5 months in the full disk of 140 \textdegree east longitude. We only
kept records of hostpots within the boundary of Australia, which reduced
to 1526080 records. Besides, a threshold (irradiance over 100 watts per
square metre) for fire power was used to filter hotspots data, which can
limit the influence of radiation from other objects. For the convention
of this algorithm, a sequence of discrete timestamps was needed. We
calculated the hourly difference between each record and the earliest
record, then rounded them to integers. The end result was a 1010794
\(\times\) 4 dataset. The four fields were the unique identifier for
each row, the longitude, the latitude and the indicator of timestamps
respectively. The code to implement this process is in ``main.R''. Read
in CSV files was done by using package \texttt{readr}. Data manipulation
was done by using package \texttt{dplyr}. High resolution Australia
vector map was obtained from package \texttt{rnaturalearth}. Operation
of geometric intersection between hotspots and Australia map was done by
using package \texttt{sf}.

\begin{Schunk}
\begin{figure}
\includegraphics[width=0.8\linewidth]{clustering_paper_files/figure-latex/hotspots-1} \caption[Hotspot locations in Victoria during2019-2020 season]{Hotspot locations in Victoria during2019-2020 season.}\label{fig:hotspots}
\end{figure}
\end{Schunk}

\hypertarget{steps}{%
\subsubsection{Steps}\label{steps}}

After the data pre-processing, this algorithm ran in a time-series
manner. It first selected entries of the first timestamps, which was the
first hour in the hotspots data. The algorithm then calculated the
matrix of pairwise geodesic distances between all points being selected.
With the geodesic distances matrix, an ``adjacent distance'' as one of
the hyperparameters in this algorithm was used to determine the
adjacency matrix. If a geodesic distance between two points was less
than the ``adjacent distance'', the corresponding entry in the adjacency
matrix would be assigned with integer 1, otherwise it would be assigned
with integer 0. Using the adjacency matrix, the algorithm then
constructed a undirected unweighted graph. For each connected component
in this graph, a unique integer was assigned as the membership. In our
hotspots data, components could be recognised as bushfires. Points in
the same component shared with the same membership. Meanwhile, the
longitude and the latitude of centroids in each component were
calculated by taking the average of longitude and the average of
latitude for all points in the corresponding component. Those centroids
along with memberships would then be recorded and labelled as active
groups. Their \texttt{active} attributes were assigned with integer 0.

For the second or later timestamps, the algorithm first

Data manipulation was done by using package \texttt{dplyr}. Geodesic
distances matrix is calculated using package \texttt{geodist}. Graph

which had the same dimension as the geodesic distances matrix.

The code implemented this algorithm is ``clustering.R''.

(Code in clustering.R does this, needs cleaning up)

\begin{enumerate}
\def\labelenumi{\arabic{enumi}.}
\tightlist
\item
  Divide hotspots by hour
\item
  Start from the first hour
\item
  Connect adjacent hotspots and active centroids (3km)
\item
  For each point, if there is a connected nearest active centroid, join
  its group
\item
  Otherwise, create a new group for each connected graph
\item
  Compute centroid for each group
\item
  Keep the group active until there is no new hotspots join the group
  within 24 hours
\item
  Repeat this process to the last hour
\end{enumerate}

\begin{Schunk}
\begin{figure}
\includegraphics[width=1\linewidth]{clustering_paper_files/figure-latex/clusters-1} \caption[Main clusters in one area over time]{Main clusters in one area over time.}\label{fig:clusters}
\end{figure}
\end{Schunk}

\hypertarget{effects-of-parameter-choices}{%
\subsubsection{Effects of parameter
choices}\label{effects-of-parameter-choices}}

\hypertarget{using-the-resulting-data}{%
\subsection{Using the resulting data}\label{using-the-resulting-data}}

\hypertarget{determining-the-ignition-point-and-time-for-individual-fires}{%
\subsubsection{Determining the ignition point and time for individual
fires}\label{determining-the-ignition-point-and-time-for-individual-fires}}

\hypertarget{tracking-fire-movement}{%
\subsubsection{Tracking fire movement}\label{tracking-fire-movement}}

\hypertarget{allocating-resources-for-future-fire-prevention}{%
\subsubsection{Allocating resources for future fire
prevention}\label{allocating-resources-for-future-fire-prevention}}

Merging data with camp sites, CFA, roads, \ldots{}

\hypertarget{summary}{%
\subsection{Summary}\label{summary}}

\hypertarget{acknowledgements}{%
\subsection{Acknowledgements}\label{acknowledgements}}

\begin{itemize}
\tightlist
\item
  The code and files to reproduce this work are at XXX
\item
  Data on hotspots can be downloaded from XXX
\end{itemize}

\bibliography{RJreferences}


\address{%
Weihao Li\\
Monash University\\
line 1\\ line 2\\
}
\href{mailto:wlii0039@student.monash.edu}{\nolinkurl{wlii0039@student.monash.edu}}

\address{%
Emily Dodwell\\
AT\&T\\
line 1\\ line 2\\
}
\href{mailto:emily@research.att.com}{\nolinkurl{emily@research.att.com}}

\address{%
Dianne Cook\\
Monash University\\
line 1\\ line 2\\
}
\href{mailto:dicook@monash.edu}{\nolinkurl{dicook@monash.edu}}

