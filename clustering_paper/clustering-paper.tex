% !TeX root = RJwrapper.tex
\title{A clustering algorithm to organize satellite hotspots data for the
purpose of tracking bushfires remotely}
\author{by Weihao Li, Emily Dodwell, Dianne Cook}

\maketitle

\abstract{%
An abstract of less than 150 words.
}

\hypertarget{introduction}{%
\subsection{Introduction}\label{introduction}}

\begin{itemize}
\tightlist
\item
  What is the data, generic structure
\item
  Lit review: Spatio-temporal clustering. Algorithms for tracking
  movement.
\item
  Bushfire literature review?
\end{itemize}

\hypertarget{algorithm}{%
\subsection{Algorithm}\label{algorithm}}

\hypertarget{data-pre-processing}{%
\subsubsection{Data pre-processing}\label{data-pre-processing}}

\begin{Schunk}
\begin{figure}
\includegraphics[width=0.8\linewidth]{clustering_paper_files/figure-latex/hotspots-1} \caption[Hotspot locations in Victoria during2019-2020 season]{Hotspot locations in Victoria during2019-2020 season.}\label{fig:hotspots}
\end{figure}
\end{Schunk}

\hypertarget{steps}{%
\subsubsection{Steps}\label{steps}}

(Code in clustering.R does this, needs cleaning up)

\begin{enumerate}
\def\labelenumi{\arabic{enumi}.}
\tightlist
\item
  Divide hotspots by hour
\item
  Start from the first hour
\item
  Connect adjacent hotspots and active centroids (3km)
\item
  For each point, if there is a connected nearest active centroid, join
  its group
\item
  Otherwise, create a new group for each connected graph
\item
  Compute centroid for each group
\item
  Keep the group active until there is no new hotspots join the group
  within 24 hours
\item
  Repeat this process to the last hour
\end{enumerate}

\begin{Schunk}
\begin{figure}
\includegraphics[width=1\linewidth]{clustering_paper_files/figure-latex/clusters-1} \caption[Main clusters in one area over time]{Main clusters in one area over time.}\label{fig:clusters}
\end{figure}
\end{Schunk}

\hypertarget{effects-of-parameter-choices}{%
\subsubsection{Effects of parameter
choices}\label{effects-of-parameter-choices}}

\hypertarget{using-the-resulting-data}{%
\subsection{Using the resulting data}\label{using-the-resulting-data}}

\hypertarget{determining-the-ignition-point-and-time-for-individual-fires}{%
\subsubsection{Determining the ignition point and time for individual
fires}\label{determining-the-ignition-point-and-time-for-individual-fires}}

\hypertarget{tracking-fire-movement}{%
\subsubsection{Tracking fire movement}\label{tracking-fire-movement}}

\hypertarget{allocating-resources-for-future-fire-prevention}{%
\subsubsection{Allocating resources for future fire
prevention}\label{allocating-resources-for-future-fire-prevention}}

Merging data with camp sites, CFA, roads, \ldots{}

\hypertarget{summary}{%
\subsection{Summary}\label{summary}}

\hypertarget{acknowledgements}{%
\subsection{Acknowledgements}\label{acknowledgements}}

\begin{itemize}
\tightlist
\item
  The code and files to reproduce this work are at XXX
\item
  Data on hotspots can be downloaded from XXX
\end{itemize}

\bibliography{RJreferences}


\address{%
Weihao Li\\
Monash University\\
line 1\\ line 2\\
}
\href{mailto:wlii0039@student.monash.edu}{\nolinkurl{wlii0039@student.monash.edu}}

\address{%
Emily Dodwell\\
AT\&T\\
line 1\\ line 2\\
}
\href{mailto:emily@research.att.com}{\nolinkurl{emily@research.att.com}}

\address{%
Dianne Cook\\
Monash University\\
line 1\\ line 2\\
}
\href{mailto:dicook@monash.edu}{\nolinkurl{dicook@monash.edu}}

