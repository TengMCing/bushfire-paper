% This is a LaTeX thesis template for Monash University.
% to be used with Rmarkdown
% This template was produced by Rob Hyndman
% Version: 6 September 2016

\documentclass{monashthesis}

%%%%%%%%%%%%%%%%%%%%%%%%%%%%%%%%%%%%%%%%%%%%%%%%%%%%%%%%%%%%%%%
% Add any LaTeX packages and other preamble here if required
%%%%%%%%%%%%%%%%%%%%%%%%%%%%%%%%%%%%%%%%%%%%%%%%%%%%%%%%%%%%%%%

\author{Weihao Li}
\title{Using remote sensing data to understand the ignition of 2019-2020
Australia bushfire season}
\studentid{28723740}
\def\degreetitle{Bachelor of Commerce (Honours)}
% Add subject and keywords below
\hypersetup{
     %pdfsubject={The Subject},
     %pdfkeywords={Some Keywords},
     pdfauthor={Weihao Li},
     pdftitle={Using remote sensing data to understand the ignition of 2019-2020
Australia bushfire season},
     pdfproducer={Bookdown with LaTeX}
}


\bibliography{thesisrefs}

\begin{document}

\pagenumbering{roman}

\titlepage

{\setstretch{1.2}\sf\tighttoc\doublespacing}

\clearpage\pagenumbering{arabic}\setcounter{page}{0}

\chapter{Statement of the topic}\label{ch:intro}

Along with the extreme heatwave in Australia 2019-2020, one of the most
devastating bushfire season in history had been witnessed. Lighting
strikes and arson were been discussed among pubilc as the main cause of
this disaster. This research will explore the sources of fire ignition
during 2019-2020 Australia bushfires season and provide a model to
predict the fire risk of neighbourhoods. Hotspots data from the JAXA's
Himawari-8 satellite and weather data from the Bureau of Meteorology of
Australia will be used in ignition identification and bushfire danger
estimation. In addition, an interactive web application embeds with
research outcomes will be built for data visualization purpose.

\section{Motivation}\label{motivation}

\textless{}left blank\textgreater{}

\section{Background and literature
review}\label{background-and-literature-review}

\textless{}left blank\textgreater{}

\section{Research aim}\label{research-aim}

The aim of this research is to use remote sensing data and apply machine
learning framework to understand the sources of bushfire ignition and
predict fire risk. There are 5 sub-objectives:

\begin{enumerate}
\def\labelenumi{\arabic{enumi}.}
\tightlist
\item
  Data integration of remote sensing data, weather data, map data and
  vegetation data.
\item
  Develop customized clustering algorithm to classify fire clusters.
\item
  Exploratory data analysis of fire clusters.
\item
  Examine source of bushfire ignitions using weather records, road map
  and recreation site locations.
\item
  Explore bushfire risk models based on weather condition, geometry
  information and fire history.
\item
  Develop an interactive shiny app to present research outcome.
\end{enumerate}

\section{Research plan}\label{research-plan}

\subsection{Data}\label{data}

In order to collect necessary datasets, two methods will be performed,
including retrieve files from servers and crawl data from websites. For
reproducibility purpose, only public data will be used in this research.
Information on the datasets can be found in Table \ref{tab:datasetinfo}.

\begin{table}[t]

\caption{\label{tab:datasetinfo}Data information}
\centering
\fontsize{9}{11}\selectfont
\begin{tabular}{llll}
\toprule
Data set name & Spatial Resolution & Temporal resolution & Time\\
\midrule
Hotspots data - JAXA’s Himawari-8 satellite & $0.02^\circ \approx 2km$ & Per 10 minutes & 2015-2020\\
Weather data - Bureau of Meteorology of \\ $\hspace{5mm}$ Australia &  & Daily & 2019-2020\\
Map - OpenStreetMap &  &  & 2020\\
Fuel layer - Australian Bureau of Agriculture \\ $\hspace{5mm}$ and Resource Economics and Sciences &  &  & 2018\\
\bottomrule
\end{tabular}
\end{table}

\subsection{Methodology}\label{methodology}

Both supervised and unsupervised learning will be implemented to reach
the research aims.

To understand the ignition of bushfires, a customized clustering
algorithm will be developed to convert hotpots data into fire history,
which will contain the starting time and coordinates of each fire. This
algorithm will mainly involve simulating fire growth, deciding fire
boundaries controlled by tolerance and assigning hotspots data to the
most probable cluster. After the clustering result being obtained, fire
history will be visualized to diagnostic the performance of the
algorithm. It will be done by comparing the behaviour of the same fire
under different sets of hyperparameters.

Exploratory data analysis of fire history and its relative factors, like
weather condition, distance to the nearest road and distance to the
nearest recreation site will be performed. Prior knowledge and featuring
engineering will be needed to fully understand the relationship. We
expect to discover relationships between the ignition of fire with these
factors, which can help us identify the cause of bushfires later on.

In order to examine the sources of fire ignition, different strategies
will be used depending on the outcome in the previous section. If the
findings from the analysis are strong and directly related to potential
sources of fire ignition, hypothesis tests will be conducted to examine
the pattern. If the evidence is weak, we will consider developing
another clustering algorithm on fire history. This algorithm will be
designed to maximize the distance between bushfire started with
different causes in a high dimensional space. A probability model then
can be built on top of it, which can provide a probabilistic answer for
the source of bushfire ignition during 2019-2020 bushfire season.

Models for predicting fire risk of neighbourhoods will be built using
raw hotspots data instead of the fire history because the hotspots data
can be considered generated from a partially observable Markov decision
process, and the underlying state is the development of the bushfire.
From low complexity models like logistic regression to high complexity
models like deep neural network will be tested.

For sharing our research outcome, a shiny app will be built and hosted
online. In addition, both static and dynamic visualization tools will be
considered using. Due to the nature of Spatio-temporal data, which has
at least 3-dimensional features, static map view without faceting can
only provide limited information. Meanwhile, faceting map view with time
will be limited by the size of caravans. Animation based map view is
computationally expensive and distracting though it provides more
information. Better ways for visualizing Spatio-temporal data will be
explored during the development.

\section{Preliminary Results}\label{preliminary-results}

\newpage

\section{irrelevant}\label{irrelevant}

\section{code}\label{code}

Included in this template is a file called \texttt{sales.csv}. This
contains quarterly data on Sales and Advertising budget for a small
company over the period 1981--2005. It also contains the GDP (gross
domestic product) over the same period. All series have been adjusted
for inflation. We can load in this data set using the following command:

\begin{Shaded}
\begin{Highlighting}[]
\NormalTok{sales <-}\StringTok{ }\KeywordTok{ts}\NormalTok{(}\KeywordTok{read.csv}\NormalTok{(}\StringTok{"data/sales.csv"}\NormalTok{)[,}\OperatorTok{-}\DecValTok{1}\NormalTok{], }\DataTypeTok{start=}\DecValTok{1981}\NormalTok{, }\DataTypeTok{frequency=}\DecValTok{4}\NormalTok{)}
\end{Highlighting}
\end{Shaded}

Any data you use in your thesis can go into the data directory. The data
should be in exactly the format you obtained it. Do no editing or
manipulation of the data outside of R. Any data munging should be
scripted in R and form part of your thesis files (possibly hidden in the
output).

\section{Figures}\label{figures}

Figure \ref{fig:deaths} shows time plots of the data we just loaded.
Notice how figure captions and references work. Chunk names can be used
as figure labels with \texttt{fig:} prefixed. Never manually type figure
numbers, as they can change when you add or delete figures. This way,
the figure numbering is always correct.

\begin{figure}
\centering
\includegraphics{thesis_files/figure-latex/deaths-1.pdf}
\caption{\label{fig:deaths}Quarterly sales, advertising and GDP data.}
\end{figure}

\section{Results from analyses}\label{results-from-analyses}

We can fit a dynamic regression model to the sales data.

If \(y_t\) denotes the sales in quarter \(t\), \(x_t\) denotes the
corresponding advertising budget and \(z_t\) denotes the GDP, then the
resulting model is:

\begin{equation}
  y_t - y_{t-4} = \beta (x_t-x_{t-4}) + \gamma (z_t-z_{t-4}) + \theta_1 \varepsilon_{t-1} + \Theta_1 \varepsilon_{t-4} + \varepsilon_t
\end{equation}

where \(\beta = 2.28\), \(\gamma = 0.97\), \(\theta_1 = NA\), and
\(\Theta_1 = -0.90\).

\section{Tables}\label{tables}

Let's assume future advertising spend and GDP are at the current levels.
Then forecasts for the next year are given in Table
\ref{tab:salesforecasts}.

\begin{table}[ht]
\begin{center}
\begin{tabular}{lrrrr}
\toprule
Point Forecast & Lo 80 & Hi 80 & Lo 95 & Hi 95 \\
\midrule
1000.2 &  947.7 & 1052.7 & 919.9 & 1080.5 \\
1013.1 &  959.3 & 1066.8 & 930.9 & 1095.3 \\
1076.7 & 1022.9 & 1130.6 & 994.4 & 1159.0 \\
1003.5 &  949.7 & 1057.4 & 921.2 & 1085.8 \\
\bottomrule
\end{tabular}
\caption{Forecasts for the next year assuming Advertising budget and GDP are unchanged.}
\label{tab:salesforecasts}
\end{center}
\end{table}

Again, notice the use of labels and references to automatically generate
table numbers. In this case, we need to generate the label ourselves.

The \texttt{knitLatex} package is useful for generating tables from R
output. Other packages can do similar things including the
\texttt{kable} function in \texttt{knitr} which is somewhat simpler but
you have less control over the result. If you use \texttt{knitLatex} to
generate tables, don't forget to include \texttt{results="asis"} in the
chunk settings.

\chapter{Exponential Smoothing}\label{sec:expsmooth}

\section{Organizing your ideas}\label{organizing-your-ideas}

Imagine you are writing for your fellow Honours students. Topics that
are well-known to them do not have to be included here. But things that
they may not know about should be included. Resist the temptation to
discuss everything you've read in the last year.

Do not organize your chapter around the papers you have read with one
section per paper. Instead, you should organize your chapters around
themes, and within each theme provide a story explaining the development
of ideas. It is usually helpful to plan out a table of contents first
with major section headings.

When you are discussing results from several papers or books, you will
need to adopt a common notation to ensure your chapter makes sense. Do
not use different notation for the same thing.

\section{Citations}\label{citations}

All citations should be done using markdown notation as shown below.
This way, your bibliography will be compiled automatically and
correctly.

Exponential smoothing was originally developed in the late 1950s
\autocites{Brown59}{Brown63}{Holt57}{Winters60}. Because of their
computational simplicity and interpretability, they became widely used
in practice.

Empirical studies by \textcite{MH79} and \textcite{Metal82} found little
difference in forecast accuracy between exponential smoothing and ARIMA
models. This made the family of exponential smoothing procedures an
attractive proposition \autocite[see][]{CKOS01}.

The methods were less popular in academic circles until \textcite{OKS97}
introduced a state space formulation of some of the methods, which was
extended in \textcite{HKSG02} to cover the full range of exponential
smoothing methods.

\appendix

\chapter{Additional stuff}\label{additional-stuff}

You might put some computer output here, or maybe additional tables.

Note that line 5 must appear before your first appendix. But other
appendices can just start like any other chapter.

\printbibliography[heading=bibintoc]



\end{document}
