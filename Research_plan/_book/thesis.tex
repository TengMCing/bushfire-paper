% This is a LaTeX thesis template for Monash University.
% to be used with Rmarkdown
% This template was produced by Rob Hyndman
% Version: 6 September 2016

\documentclass{monashthesis}

%%%%%%%%%%%%%%%%%%%%%%%%%%%%%%%%%%%%%%%%%%%%%%%%%%%%%%%%%%%%%%%
% Add any LaTeX packages and other preamble here if required
%%%%%%%%%%%%%%%%%%%%%%%%%%%%%%%%%%%%%%%%%%%%%%%%%%%%%%%%%%%%%%%

\author{Weihao Li}
\title{Using Remote Sensing Data to Understand Fire Ignition during the
2019-2020 Australia Bushfire Season}
\studentid{28723740}
\def\degreetitle{Bachelor of Commerce (Honours)}
% Add subject and keywords below
\hypersetup{
     %pdfsubject={The Subject},
     %pdfkeywords={Some Keywords},
     pdfauthor={Weihao Li},
     pdftitle={Using Remote Sensing Data to Understand Fire Ignition during the
2019-2020 Australia Bushfire Season},
     pdfproducer={Bookdown with LaTeX}
}


\bibliography{thesisrefs}

\begin{document}

\pagenumbering{roman}

\titlepage

{\setstretch{1.2}\sf\tighttoc\doublespacing}

\clearpage\pagenumbering{arabic}\setcounter{page}{0}

\chapter{Statement of the topic}\label{ch:intro}

Along with the extreme heatwave in Australia 2019-2020, one of the most
devastating bushfire season in history had been witnessed. Lighting
strikes and arson were been discussed among pubilc as the main cause of
this disaster. This research will explore the methods of fire ignition
during 2019-2020 Australia bushfires season and provide a model to
predict the fire risk of neighbourhoods. Hotspots data from the JAXA's
Himawari-8 satellite and weather data from the Australian Bureau of
Meteorology will be used in ignition identification and bushfire danger
estimation. In addition, an interactive web application embeds with
research outcomes will be built for data visualization purpose.

\section{Motivation}\label{motivation}

In Australia, numerous occasions of bushfires cause losses of properties
and life every year. Recorded since Hobart bushfires in 1967, the
insurance claimed for building losses was greater than \$A10 million
(McAneney, Chen \& Pitman, 2009) McAneney, Chen and Pitman (2009) show
in their research that the average number of buildings had been
destroyed by bushfire per year is 84, accounting for 20\% of total
building losses in hazard events.

The Australian bushfire season in 2019-2020, compared with other major
bushfires in history, had a severer impact on environment and
properties. According to the Parliament of Australia report (2020) on
this devastating bushfire season, 3094 houses had been destroyed during
this crisis and burnt land was over 17M hectares. These two figures are
the highest in history. Fortunately, only 33 people including
firefighters died comparing to 173 in Black Saturday and 47 in Ash
Wednesday (Parliament of Australia, 2020).

Therefore, understanding what caused the 2019-2020 Australian bushfires
is extremely important which may provide us with information for future
legislation and fire risk management.

Another motivation for us to conduct this research is the difficulty of
finding the cause of bushfire in Australia. Beale and Jones (2009) state
in their research that only 58.9\% of the cause of fires is known. In
the known cases, the percentage of deliberate and suspicious ignitions
is around 50\%. Besides, 35\% of known cases are caused by accidents and
6\% of known cases are caused by nature, for example, lightning. Provide
probability information on the cause of bushfires may help investigators
in practice.

\section{Research aim and questions}\label{research-aim-and-questions}

\textless{}in progress\textgreater{}

This research aims to answer the following questions:

\begin{enumerate}
\def\labelenumi{\arabic{enumi}.}
\item
  How to clustering bushfires from hotspots data?
\item
  How to find the ignition methods during 2019-2020 Australia bushfire
  season?

  \begin{enumerate}
  \def\labelenumii{\alph{enumii}.}
  \tightlist
  \item
    How to identify the cause of ignition without classes labels?
  \item
    What is the characteristics of 2019-2020 bushfires comparing to
    historical bushfires?
  \end{enumerate}
\item
  How to model fire risk of neighbourhoods?

  \begin{enumerate}
  \def\labelenumii{\alph{enumii}.}
  \tightlist
  \item
    Is fire indexes useful for fire prediction?
  \item
    Will likelihood of ignitions increase near the campping site and
    road?
  \item
    Are nonparametric models better than Generalized linear model and
    Generalized additive model in fire risk prediction?
  \item
    How to take into account the temporal patterns and spatial patterns
    in fire risk modelling?
  \item
    How to apply regularization and add noise in fire risk modelling?
  \end{enumerate}
\end{enumerate}

\section{Research plan}\label{research-plan}

This reasearch will start from data collection. Various datasets will be
collected for this research, and the primary datasets will be hotspots
data, weather data and map data.

To understand the ignition of bushfires, a customized clustering
algorithm will be developed to convert hotpots data into fire history,
which will contain the starting time and coordinates of each fire. This
algorithm will mainly involve simulating fire growth, deciding fire
boundaries controlled by tolerance and assigning hotspots data to the
most probable cluster. After the clustering result being obtained, fire
history will be visualized to diagnostic the performance of the
algorithm. It will be done by comparing the behaviour of the same fire
under different sets of hyperparameters.

Exploratory data analysis of fire history and its relative factors, like
weather condition, distance to the nearest road and distance to the
nearest recreation site will be performed. Prior knowledge and featuring
engineering will be needed to fully understand the relationship. We
expect to discover relationships between the ignition of fire with these
factors, which can help us identify the cause of bushfires later on.

In order to examine the sources of fire ignition, different strategies
will be used depending on the outcome in the previous section. If the
findings from the analysis are strong and directly related to potential
sources of fire ignition, hypothesis tests will be conducted to examine
the pattern. If the evidence is weak, we will consider developing
another clustering algorithm on fire history. This algorithm will be
designed to maximize the distance between bushfire started with
different causes in a high dimensional space. A probability model then
can be built on top of it, which can provide a probabilistic answer for
the cause of bushfires during 2019-2020 bushfire season.

Models for predicting fire risk of neighbourhoods will be built using
raw hotspots data instead of the fire history because the hotspots data
can be considered generated from a partially observable Markov process,
and the underlying state is the development of the bushfire. From low
complexity models like logistic regression to high complexity models
like random forest will be tested.

For sharing our research outcome, a shiny app will be built and hosted
online. In addition, both static and dynamic visualization tools will be
considered using. Due to the nature of Spatio-temporal data, which has
at least 3-dimensional features, static map view without faceting can
only provide limited information. Meanwhile, faceting map view with time
will be limited by the size of caravans. Animation based map view is
computationally expensive and distracting though it provides more
information. Better ways for visualizing Spatio-temporal data will be
explored during the development. The potential product will be an
interactive map view with triggers to transform data and manipulate the
aesthetics specifications.

\section{Scope of research}\label{scope-of-research}

In this research, we will primarily focus on Victoria but eventually
extend the results throughout Australia. The study area of this reserach
will be limited to bushfires since 2000, especially 2019-2020 bushfires.
What else?

\textless{}in progress\textgreater{}

\chapter{Literature Review}\label{literature-review}

Existing researches in bushfires modelling can be divided into two main
categories, one is simulation modelling, another is analytical
modelling.

In simulation modelling, Keane et al. (2004) have attempted to use
landscape fire succession models (LFSMs) to model fire behaviour
including fire ignition and fire spread. They are a group of spatial
simulation models which taking into account fire and vegetation
dynamics. Similarly, Bradstock et al. (2012) used FIRESCAPE model which
mainly involved simulating fire behaviours with fuel and weather
conditions. Simulation methods are cost-effective and time-effective in
modelling bushfires (Clarke et al., 2019). However, Clarke et al. (2019)
also stated in their research that ignition likelihood is not well
discussed and considered in these models. Besides, these methods
seldomly address ignition types of bushfires which we are interested in.
Therefore, simulation modelling will not be considered in our bushfire
ignition research.

Alternatively, analytical modelling is a more popular way to build
bushfires models. In analytical modelling, the general framework for
analysing bushfires ignitions is generalised additive model (GAM).
Bates, McCaw and Dowdy (2018) used it for predicting the number of
lightning ignitions. Some studies include a logit link to extend the
model for bushfires ignition likelihood prediction (Read, Duff and
Taylor, 2018; Zhang, Lim and Sharples, 2017). Mixed-effects had also
been considered for incorporating spatial and weather factors (Duff,
Cawson and Harris, 2018). Simpler models were been used in this field,
like multiple linear regression, negative binomial regression and
generalised logistic regression (Cheney, Gould, McCaw and Anderson,
2012; Plucinski, McCaw, Gould and Wotton, 2014; Collins, Price and
Penman, 2015). Particularly, instead of using a model, some researches
performed statistical testing and exploratory data analysis to test
certain hypothesises of bushfires (Miller et al., 2017; Dowdy, Fromm and
McCarthy, 2017).

Common covariates for ignitions analysis are weather conditions,
vegetation types, topographic information and anthropogenic variables.
In addition, various indexes had been used in modelling. Some studies
choose to use index variables developed by McArthur such as Forest Fire
Danger Index (Clarke et al., 2019; Read et al., 2018), while others
choose to use indexes developed by Canadian Forestry Service such as
Canadian Fire Weather Index and Drought Code (Plucinski et al., 2014).
We doubt that these indexes are irrelevant with fire ignition prediction
because they are extracted from weather and vegetation information.
However, it may help improve our model performance because it can be
viewed as features generated from feature engineering. We will test if
index variables are significant in our research.

Although numerous studies for ignitions analysis have applied
semiparametric and parametric methods, little analytics attention has
been paid to more complex model like tree-based model, support vector
machine and artificial neural network. These tools are well developed in
machine learning, which will be considered in this research.

Most of the existing works of bushfire ignition analysis focus on a
certain area such as south-eastern Australia (Clarke et al., 2019), or a
certain state such as Victoria (Read, Duff \& Taylor, 2018). Little of
existing works have applied prediction across Ausralia.

\chapter{Data collection and
processing}\label{data-collection-and-processing}

There are numerous open source data sets available that are collated to
provide the data to address the research questions. The main data
resource that differs from that used in the literature is the satellite
data that records hotspots. The next sections describe how the data is
accessed, and and pre-processed for later analysis.

\section{Sources}\label{sources}

Table \ref{tab:datasetinfo} summarises the data sources.

\subsection{Hotspots}\label{hotspots}

Hotspot data is downloaded from the JAXA's Himawari-8 satellite. This
satellite is positioned in geostationary orbit at 140 degrees east
longitude, and the revisit period is 10-minute. Its management system -
JAXA's P-Tree system, provides WildFire observation product with 2km
spatial resolution. (P-Tree System, 2020). Details on how to download
this data are provided by {[}ozjim post{]}.

\subsection{Weather}\label{weather}

Weather data were collected from the Australian Bureau of Meteorology,
by using an R package - Bomrang (Adam, Mark, Hugh \& Keith, 2020). Due
to the limitation of APIs provided by the package, we crawled data from
BOM's website for extra information.

\subsection{Fuel layer}\label{fuel-layer}

To characterise the fuel we used forest of Australia (2018) from the
Australian Bureau of Agriculture and Resource Economics and Sciences. It
is a fuel layer contains the vegetation information across Australia.

\subsection{Fire origins}\label{fire-origins}

Fire origins are the existing records of historical bushfires ignition
which is downloaded from the Department of Environment, Land, Water
Planning. This dataset is used to understand the characteristic of each
type of ignition. It helps us to identify the causes of 2019-2020
bushfires.

\begin{table}[t]

\caption{\label{tab:datasetinfo}Data information}
\centering
\fontsize{9}{11}\selectfont
\begin{tabular}{llll}
\toprule
Data set name & Spatial Resolution & Temporal resolution & Time\\
\midrule
Hotspots data - JAXA’s Himawari-8 satellite & $0.02^\circ \approx 2km$ & Per 10 minutes & 2015-2020\\
Weather data - Australian Bureau of Meteorology &  & Daily & 2019-2020\\
Map - OpenStreetMap & 2m &  & 2020\\
Fuel layer - Australian Bureau of Agriculture \\ $\hspace{5mm}$ and Resource Economics and Sciences & 100m &  & 2018\\
Victorian CFA fire stations - Department of Environment, \\ $\hspace{5mm}$Land, Water $\&$ Planning & 20m &  & 2020\\
\addlinespace
Victorian Recreation sites - Department of Environment, \\ $\hspace{5mm}$Land, Water $\&$ Planning & 10m &  & 2020\\
Fire Origins - Department of Environment, \\ $\hspace{5mm}$Land, Water $\&$ Planning & 100m &  & 1972-2019\\
\bottomrule
\end{tabular}
\end{table}

\section{Pre-processing}\label{pre-processing}

\subsection{Data types}\label{data-types}

The hotspot and weather data is csv format, which can be processed
generally with the \texttt{tidyverse} tools (Wickham et al., 2020) in R.
Fuel layer and other map data are presented as geospatial objects, which
can be processed using the tools in the \texttt{sf} (Pebesma, 2020).

\subsection{hotspots clustering}\label{hotspots-clustering}

Hotpots are point data obtained by light detection from the satellite.
They are snapshots of bushfires every 10 minutes. However, just like it
is hard to identify a single plant from the top view of a grassland,
information about individual bushfire can not be directly derived from
the raw hotspot data. In order to understand the ignition time and
ignition place of each bushfire, we need to divide hotspots into
clusters and trace the growth of each cluster. To preprocess the
hotspots data, we selected the observations in Australia from the full
disk. Meanwhile, hotspots with irradiance under 100 watt per square
metre will be deleted. We restricted our study to hotspots that have
significant firepower. An hour id has been assigned to each observation
range from 1 to T, represents the relative time the hotspot being
observed. On top of the tidy hotspots data, a clustering algorithm was
developed to identify fire clusters. Details about the algorithm can be
found in table \ref{tab:clustering}.

By using this algorithm, we assigned each hotspot a cluster membership
which we called \emph{fire\_id}. Meanwhile, we recorded the
characteristics of each cluster including its centroid, starting time,
ending time and movement. The inspiration behind this algorithm is the
behaviour of real world bushfire which can be summarised as two
hyperparameters, the distance of spread in each hour and the lifetime of
fire since last observed, represented by \(r_0\) and \(t_0\)
respectively. These two hyperparameters have been used to determine if a
new hotspot belongs to an existing or new cluster.

\normalfont

\begin{table}
\caption{\label{tab:clustering}A clustering algorithm for hotspots}
\begin{align*}
&\rule{150mm}{0.5mm}\\[-1\jot]
&\textbf{Algorithm 1 Hotspots clutering}\\[-1\jot]
&\rule{150mm}{0.5mm}\\[-1\jot]
&\textbf{input: }~~~~\text{Hotspots dataset H : (Hour}\textunderscore \text{id}^{(n)} \text{, Coordinates}^{(n)} \text{), n = 1, 2, ... N}\\[-1\jot]
&~~~~~~~~~~~~~~~~\text{An empty dataset F : (Fire\textunderscore id}^{(m)} \text{, Coordinates}^{(m)} \text{, Active}^{(m)} \text{), m = 1, 2, ...}\\[-1\jot]
&~~~~~~~~~~~~~~~~\text{An empty vector K} \in \mathbb{N}_1^n\\[-1\jot]
&~~~~~~~~~~~~~~~~\text{A distance hyperparameter }r_0 \in \mathbb{R}^+\\[-1\jot]
&~~~~~~~~~~~~~~~~\text{A time hyperparameter }t_0 \in \mathbb{N}^+\\[-1\jot]
&\textbf{output: }~~\text{A vector K} \in \mathbb{N}_1^n~\text{contains memberships of hotspots}\\[-1\jot]
&~~~~~~~~~~~~~~~~\text{A dataset F contains fire clusters information including memberships, latest}\\[-1\jot]
&~~~~~~~~~~~~~~~~\text{centroids and time from last updated}\\[-1\jot]
&~~1:~~\text{select subset }H_c \in \text{H where Hour\textunderscore id == 1}\\[-1\jot]
&~~2:~~\text{calculate distance matrix D for Coordinates in }H_c \\[-1\jot]
&~~3:~~\text{assign 1 to a zero adjacency matrix A for where D} \leq r_0~~//~~\text{hotspots with relative}\\[-1\jot]
&~~~~~~~~~\text{distance less or equal to }r_0~\text{will be considered belong to the same cluster} \\[-1\jot]
&~~4:~~\text{create undirected unweighted graph G from A}\\[-1\jot]
&~~5:~~\text{record memberships of G to K}\\[-1\jot]
&~~6:~~\text{record clusters classes to Fire\textunderscore id and record clusters centroids to Coordinates of F}\\[-1\jot]
&~~7:~~\text{set Active in F to }t_0~~//~~\text{Active clusters are fire being observed in the last }t_0~\text{hour}\\[-1\jot]
&~~8:~~\textbf{for}~~\text{hour = 2, ... T}~~\textbf{do} \\[-1\jot]
&~~9:~~~~~~~~\text{let Active -1 and select subset }F_c \in \text{F where Active} \geq 0\\[-1\jot]
&10:~~~~~~~~\text{select subset }H_c \in \text{H where Hour\textunderscore id == hour}\\[-1\jot]
&11:~~~~~~~~\text{append Coordinates from }F_c~\text{to}~H_c\\[-1\jot]
&12:~~~~~~~~\text{repeat step 2 - 4}\\[-1\jot]
&12:~~~~~~~~\textbf{for}~~h_i = \text{each hotspot in }H_c~~\textbf{do}\\[-1\jot]
&13:~~~~~~~~~~~~~~\textbf{if}~~h_i~\text{share the same membership as one of active clusers in }F_c~~\textbf{then}\\[-1\jot]
&14:~~~~~~~~~~~~~~~~~~~~\text{copy the corresponding Fire\textunderscore id of the nearest active cluser to K}\\[-1\jot]
&15:~~~~~~~~~~~~~~\textbf{else}~~\text{copy the membership from G to K}\\[-1\jot]
&16:~~~~~~~~~~~~~~\textbf{end if}\\[-1\jot]
&17:~~~~~~~~\textbf{end for}\\[-1\jot]
&18:~~~~~~~~\text{update F for clusters involed in current timestamp and reset corresponding}\\[-1\jot]
&~~~~~~~~~~~~~~~\text{Active to }t_0\\[-1\jot]
&19:~~\textbf{end for}\\[-1\jot]
&\rule{150mm}{0.5mm}
\end{align*}
\end{table}

\section{Compiled data}\label{compiled-data}

The end result of the data pre-processing is a set of data tables, that
are uniquely identify by individual primary key and related by a set of
foreign keys. Figure \ref{fig:ERD} shows these tables and the key which
allow information to be compared across tables. Technically, this figure
is called a conceptual entity relationship diagram. It will be useful
for describing the data model to be used in both modelling and web
interface for communicating fire risk.

\begin{figure}
\centering
\includegraphics{figures/Shiny_app_data_Conceptual_ERD.jpeg}
\caption{Entity relationship diagram illustrating the relational tables
of the compiled data. Tables correspond to processed hotspot data,
weather, local facilities like CFA sites. This data structure is useful
for the data modeling and web app development. \label{fig:ERD}}
\end{figure}

\chapter{Exploratory data analysis}\label{exploratory-data-analysis}

\section{Overview of 2019-2020 Australia bushfire
season}\label{overview-of-2019-2020-australia-bushfire-season}

\begin{figure}
\centering
\includegraphics[width=5.20833in]{figures/number_of_ignitions.jpg}
\caption{The grid map illustrating the general situation of Victoria
during 2019-2020 Australia bushfire season. The severest time for
Victoria was December when the massive amount of bushfires ignited in
Eastern Victoria. Places like East Gippsland suffered from this
devastating crisis.}
\end{figure}

\section{weather condition}\label{weather-condition}

\begin{figure}
\centering
\includegraphics[width=5.20833in]{figures/current_weather_condition.jpg}
\caption{The densities plot of Weather conditions illustrating the
relationship between bushfire ignitions and maximum temperature, minimum
temperature and global solar exposure respectively. Bushfires ignition
tends to happen when the maximum temperature and the global solar
exposure are higher than normal. This relationship is useful for
ignition prediction. \label{fig:weather-conditions}}
\end{figure}

\begin{figure}
\centering
\includegraphics[width=5.20833in]{figures/history_weather_condition.jpg}
\caption{The densities plot of weather conditions illustrating the
relationship between historical lightning-ignition bushfires and
2019-2020 bushfires. Historical lightning-ignition bushfires occurred
under higher temperature but lower global solar exposure than 2019-2020
bushfires. It tells us that 2019-2020 bushfires are probabaly not only
caused by lightning.}
\end{figure}

\begin{figure}
\centering
\includegraphics[width=5.20833in]{figures/his_lightning.jpg}
\caption{The densities plot of weather conditions illustrating the
relationship between lightning-ignition bushfires and other casued
bushfires. Lightning-ignition bushfires occurred under high temperature
but not neccssary high global solar exposure. It is useful for ignition
type classification.}
\end{figure}

\section{Distance to the nearest road and recreation
site}\label{distance-to-the-nearest-road-and-recreation-site}

\begin{figure}
\centering
\includegraphics[width=5.20833in]{figures/dist_to_camp_and_road.jpg}
\caption{The densities plot of distance to the nearest recreation site
and the nearest road shows the importance of these two variables.
Simulation draws are random simulated from Victoria, which assumes
bushfires can occur anywhere. The observed draws of distance to the
nearest road is not significantly different with simulated draws, which
suggests that this variable is not useful for ignition type
classification in most of the time. Observed bushfires are closer to the
camping site, which implies camping sites have higher fire risk.}
\end{figure}

\begin{figure}
\centering
\includegraphics[width=5.20833in]{figures/dist_to_camp_waffle.jpg}
\caption{The waffle plot of distance of fire ignitions to the nearest
recreation site shows the possibility of human lighted bushfires. Only
1\% of ignitions within the 1km radius of recreation sites, and only 3\%
within 3km radius. The chance of human lighted fire is high.}
\end{figure}

\begin{figure}
\centering
\includegraphics[width=5.20833in]{figures/dist_to_road_waffle.jpg}
\caption{The waffle plot of distance of fire ignitions to the nearest
road illustrating the possibility of human ingnited. 13\% of the
distance less than 100m, which suggests high probability of accidentally
or deliberately lighting. Notice that 50\% of fires are very close to
road (less than 500m), which means they are accessible for
firefighters.}
\end{figure}

\section{Fire cluster movement}\label{fire-cluster-movement}

\begin{figure}
\centering
\includegraphics[width=5.20833in]{figures/fire_mov.jpg}
\caption{The movement path and hotspots map shows the bushfire behaviour
of bushfire ``454'' and bushfire ``1420'', which is two clusters from
the results of clustering algorithm. The triangle is the ignition point
of bushfire and the rectangle is the extinguish point. The black line is
the movement of bushfire centeroid. points with red color and larger
size are hotspots with higher firepower which is measured by watt per
square metre. We can know that bushfires ignited with low firepower, and
will getting hotter when time pass, then finally extinguish with not
enough fuel.}
\end{figure}

Which plot should I keep? What other plots should I include in this
chapter?

\chapter{Modelling}\label{modelling}

\textless{}plan\textgreater{}

In this research, two type of models will be developed and tested. One
is ignition method prediction model, which is used for analysing
ignition type of 2019-2020 bushfires and predicting the causes of future
bushfires. Another is fire risk model. It will learn the features of
hotspots, and predict the probability of hotspots occurrence at a
paticular time and place. The following sections will discuss the plan
to develop these two models.

\section{Predicting ignition method}\label{predicting-ignition-method}

\textless{}plan\textgreater{}

To analysis and predict ignition method of bushfires, historical causes
of bushfires will be used as traning data. The modelling problem will be
a multiclass classification problem. Incorporating vegetation,
topographic, anthropogenic and climate information as independent
variables, a group of models will be tested. These include parametric
models like generalized linear model, semiparametric models like
generalized additive model and nonparametric models like random forest,
support vector machine and artificial neural network.

Statistical inference will be performed for testing the differences
between human lighted bushfires and ligntning-ignition bushfires.

``Differences between historical bushfires and 2019-2020 bushfires will
be taken into account?''

\section{Modelling fire risk}\label{modelling-fire-risk}

The framework of fire risk modelling is simialr to ignition method
prediction. It will still be a classification problem, but a binary
classification problem instead. The aim of this model is to predict the
probability of bushfire ignition in a given grid at a given time. The
grid will be designed to fit the shape of Victoria and has spatial
resolution equal to 50km. The training data will be the hotspots data
instead of the bushfire history. Potential models will be the same as
ignition method prediction.

\textless{}plan\textgreater{}

\chapter{Timeline}\label{timeline}

The research plan for this semseter can be found in Table
\ref{tab:timeline1}. Future research plan can be found in Table
\ref{tab:timeline2}.

\begin{table}[!h]

\caption{\label{tab:timeline1}Research plan till week 9}
\centering
\begin{tabular}{ll}
\toprule
Timeline & Tasks\\
\midrule
Week 2 & Geographic data background reading\\
Week 3 & Collect Remote sensing data (JAXA himawari-8 satellite) and\\
 & explore BOM weather data APIs (Bomrang)\\
Week 4 & Collect Road Map (OpenStreetMap) and\\
 & read articles in SpatioTemporal data visualization and modelling\\
\addlinespace
Week 5 & Develop clustering algorithm for remote sensing data\\
Week 6 & Test diferent hyperparameters for clusetring\\
Week 7 & Exploratory data analysis on fire clusters and data integration\\
Week 8 & Feature planning for the shiny app\\
week 9 & Write research proposal and prepare the first presentation\\
\bottomrule
\end{tabular}
\end{table}

\begin{table}[!h]

\caption{\label{tab:timeline2}Research plan since June}
\centering
\begin{tabular}{ll}
\toprule
Timeline & Tasks\\
\midrule
June - July & Modelling fire ignition and fire risk\\
August & Consolidate findings and create mockups of the shiny app\\
September & Develop the shiny app and perform different levels of testing\\
October & Write thesis and prepare the second presentation\\
\bottomrule
\end{tabular}
\end{table}

\chapter{Supplementary materials}\label{supplementary-materials}

\chapter{Biblography}\label{biblography}

\appendix

\chapter{Additional stuff}\label{additional-stuff}

You might put some computer output here, or maybe additional tables.

Note that line 5 must appear before your first appendix. But other
appendices can just start like any other chapter.

\printbibliography[heading=bibintoc]



\end{document}
